\documentclass[12pt,a4paper]{article}

%%% Packages %%%

%\usepackage[frenchb]{babel}
\usepackage[utf8]{inputenc}
\usepackage[T1]{fontenc}
\usepackage{lmodern}
\usepackage{amsmath}
\usepackage{amsfonts}
\usepackage{amssymb}
\usepackage{graphicx}
\usepackage{float}
%\usepackage{algorithm}
%\usepackage{algpseudocode}
\usepackage{setspace}
\usepackage[textfont={it}]{caption}

%%% Environnement bigcenter : permet de centrer des images plus larges que \linewidth %%%
\makeatletter
\newskip\@bigflushglue \@bigflushglue = -100pt plus 1fil
\def\bigcenter{\trivlist \bigcentering\item\relax}
\def\bigcentering{\let\\\@centercr\rightskip\@bigflushglue
\leftskip\@bigflushglue
\parindent\z@\parfillskip\z@skip}
\def\endbigcenter{\endtrivlist}
\makeatother

\newcommand{\reporttitle}{\vspace{3mm} MPRO-ECMA}
\newcommand{\reportauthor}{Paulin \textsc{Jacquot}\\Roxane \textsc{Delpeyrat}}
\newcommand{\reportsubject}{Projet 2015-2016}
\newcommand{\HRule}{\rule{\linewidth}{0.5mm}}
\setlength{\parskip}{1ex}

\newcommand{\bb}[1]{\mathbb{#1}}
\renewcommand{\cal}[1]{\mathcal{#1}}

%%% Début du rapport %%%%%%%%%%%%%%%%%%%%%%%%%%%%%%%%%%%%%%%%%%%%%%%%%%%%%%%
\begin{document}

%%% Page de garde %%%%%%%%%%%%%%%%%%%%%%%%%%%%%%%%%%%%%%%%%%%%%%%%%%%%%%%%%%
\begin{titlepage}
	\begin{center}
		\begin{minipage}[c]{0.50\textwidth}
			\begin{flushright}
				% logo 1
			\end{flushright}
		\end{minipage}
		\hfill
		\begin{minipage}[c]{0.50\textwidth}
			\begin{flushleft}
				% logo 2
			\end{flushleft}
		\end{minipage}
		
		\vspace{2cm}
		
		\textsc{\Large \reportsubject}\\[0.5cm]
		\HRule \\[0.4cm]
		{\huge \bfseries \reporttitle}\\[0.4cm]
		\HRule \\[1.5cm]
		\begin{minipage}[t]{0.35\textwidth}
			\begin{flushleft} \large
				\emph{Auteurs :}\\
					\reportauthor
			\end{flushleft}
		\end{minipage}
		\vfill
		{\large \today}
	\end{center}
\end{titlepage}


%%% E1 %%%%%%%%%%%%%%%%%%%%%%%%%%%%%%%%%%%%%%%%%%%%%%%%%%%%%%%
\paragraph{Exercice 1\\}

1.1 On utilise des variables $x_{ij}$ pour $(i,j) \in M $, valant $1$ SSI la maille $(i,j)$ est selectionnée. Le programme s'écrit alors :

\begin{align}
\label{obj} max & \sum_{(ij)\in M} x_{ij} \\
s.c. \label{critere} \ & \dfrac{\sum H_{ij}^pC_{ij}^p x_{ij}}{\sum C_{ij}^p x_{ij}}+ \dfrac{\sum H_{ij}^a C_{ij}^a x_{ij}}{\sum C_{ij}^a x_{ij}}\geq 2 \\
& x_{ij}\in \{0,1\}
\end{align}

\vspace{0.5cm}

1.2 Pour linéariser la contrainte fractionnaire (\ref{critere}), on utilise les variables :

\begin{align*}
y_{ij}=\dfrac{1}{\sum C_{ij}^p x_{ij}} \ \ & \ \  z_{ij} =\dfrac{1}{\sum C_{ij}^a x_{ij}}
\end{align*}

ce qui donne les contraintes quadratiques :

\begin{align} \label{Cq}
 \sum C_{ij}^p y \cdot x_{ij} = 1 \ \ & \ \   \sum C_{ij}^a z \cdot x_{ij} = 1
\end{align}

On linéarise ensuite ces contraintes. Pour cela, introduisons les quantités :

\begin{align*}
M^{p} = \dfrac{1}{\underset{(i,j) \in M }{min} C_{ij}^p} \ \ & \ \ M^{a} = \dfrac{1}{\underset{(i,j) \in M }{min} C_{ij}^a} 
\end{align*}

bien définies car les coefficients $C^p $ et $C^a$ sont strictement positifs pour tout $(ij) \in M$.

En posant $u_{ij} = x_{ij} \cdot y$ et $v_{ij} = x_{ij} \cdot z$, les contraintes (\ref{Cq}) sont alors équivalentes à :

\begin{align} \label{Clin1}
\sum_{(i,j) \in M} & C^p_{ij}\cdot u_{ij} =1 \ \ & \ \ \sum_{(i,j) \in M} & C^a_{ij}\cdot v_{ij} =1 \  & \\
\label{Clin2} u_{ij} & \leq x_{ij} \cdot M^p \ \ & \ \  v_{ij} &\leq x_{ij} \cdot M^a, &\ \forall (i,j) \in M  \\
\label{Clin3} u_{ij} & \leq y  \ \ & \ \  v_{ij} & \leq z, &\ \forall (i,j) \in M  \\
\label{Clin4} u_{ij} & \geq (x_{ij}-1) \cdot M^p +y\ \ & \ \ v_{ij} & \geq (x_{ij}-1) \cdot M^a +z, & \forall (i,j) \in M \\
\label{Clin5} u_{ij} & \geq 0 \ \ & \ \  v_{ij} & \geq 0, &\ \forall (i,j) \in M  
\end{align}

La contrainte (2) se réécrit également de façon linéaire :

\begin{equation} \label{critereLin}
\sum H_{ij}^p C_{ij}^p u_{ij} + \sum H_{ij}^aC_{ij}^a v_{ij} \geq 2
\end{equation}

Le programme linéaire s'obtient avec les contraintes (\ref{critereLin}) et \ref{Clin1}, \ref{Clin2},\ref{Clin3},\ref{Clin4},\ref{Clin5} et la même fonction objectif : $$max \sum_{(ij)\in M} x_{ij}$$

\vspace{1cm}


%%% E1 %%%%%%%%%%%%%%%%%%%%%%%%%%%%%%%%%%%%%%%%%%%%%%%%%%%%%%%
\paragraph{Exercice 2\\}

Définissons, pour chaque $h\in [| 0,n^2 |] $, les variables binaires $ l_{ijh} \forall (i,j) \in M$.
On modélise alors la connexité comme le suggère l'énoncé : il existe une et une seule maille ``racine''  de hauteur $h=0$ . Ensuite, chaque maille $(ij)$ sélectionnée se voit attribuer une hauteur $h$ (et alors $l_{ijh}=1$ ) et une maille est selectionnée avec hauteur $h>0$ si une de ses voisines est selectionnée avec hauteur $h-1$.

Pour toute maille selectionnée $(i,j)$, il existe donc un chemin empruntant des mailles selectionnées jusqu'à la maille racine de hauteur 0. La solution est donc étoilée par rapport à cette maille racine, donc connexe.

Les contraintes s'écrivent donc de la manière suivante : 

\begin{align} \label{Cconnex}
\sum_{(ij)\in M} l_{ij0} &= 1 \ \text{(une et une seule racine)} \\
\sum_{h=0}^{n^2} l_{ijh} &= x_{ij} , \ \forall (i,j) \in M \\
l_{ijh+1} & \leq l_{i-1jh} +l_{i+1jh} +l_{ij-1h} +l_{ij+1h}, \forall (i,j) \in M, \ h\in [|0, n^2 -1|] \\
\l_{ijh} & \in \{0,1\} , \ \forall (i,j) \in M, \forall h\in [|0, n^2|] 
\end{align}

Cela représentant un très grand nombre de variables et de contraintes (en $\mathcal{O}(n^4) $ ), on pourra ajouter les contraintes au fur et à mesure, seulement si elles sont violées.

\end{document}